% Options for packages loaded elsewhere
\PassOptionsToPackage{unicode}{hyperref}
\PassOptionsToPackage{hyphens}{url}
\PassOptionsToPackage{dvipsnames,svgnames,x11names}{xcolor}
%
\documentclass[
  12pt,
  number]{article}

\usepackage{amsmath,amssymb}
\usepackage{lmodern}
\usepackage{iftex}
\ifPDFTeX
  \usepackage[T1]{fontenc}
  \usepackage[utf8]{inputenc}
  \usepackage{textcomp} % provide euro and other symbols
\else % if luatex or xetex
  \usepackage{unicode-math}
  \defaultfontfeatures{Scale=MatchLowercase}
  \defaultfontfeatures[\rmfamily]{Ligatures=TeX,Scale=1}
\fi
% Use upquote if available, for straight quotes in verbatim environments
\IfFileExists{upquote.sty}{\usepackage{upquote}}{}
\IfFileExists{microtype.sty}{% use microtype if available
  \usepackage[]{microtype}
  \UseMicrotypeSet[protrusion]{basicmath} % disable protrusion for tt fonts
}{}
\makeatletter
\@ifundefined{KOMAClassName}{% if non-KOMA class
  \IfFileExists{parskip.sty}{%
    \usepackage{parskip}
  }{% else
    \setlength{\parindent}{0pt}
    \setlength{\parskip}{6pt plus 2pt minus 1pt}}
}{% if KOMA class
  \KOMAoptions{parskip=half}}
\makeatother
\usepackage{xcolor}
\usepackage[top=30mm,left=20mm,heightrounded]{geometry}
\setlength{\emergencystretch}{3em} % prevent overfull lines
\setcounter{secnumdepth}{5}
% Make \paragraph and \subparagraph free-standing
\ifx\paragraph\undefined\else
  \let\oldparagraph\paragraph
  \renewcommand{\paragraph}[1]{\oldparagraph{#1}\mbox{}}
\fi
\ifx\subparagraph\undefined\else
  \let\oldsubparagraph\subparagraph
  \renewcommand{\subparagraph}[1]{\oldsubparagraph{#1}\mbox{}}
\fi


\providecommand{\tightlist}{%
  \setlength{\itemsep}{0pt}\setlength{\parskip}{0pt}}\usepackage{longtable,booktabs,array}
\usepackage{calc} % for calculating minipage widths
% Correct order of tables after \paragraph or \subparagraph
\usepackage{etoolbox}
\makeatletter
\patchcmd\longtable{\par}{\if@noskipsec\mbox{}\fi\par}{}{}
\makeatother
% Allow footnotes in longtable head/foot
\IfFileExists{footnotehyper.sty}{\usepackage{footnotehyper}}{\usepackage{footnote}}
\makesavenoteenv{longtable}
\usepackage{graphicx}
\makeatletter
\def\maxwidth{\ifdim\Gin@nat@width>\linewidth\linewidth\else\Gin@nat@width\fi}
\def\maxheight{\ifdim\Gin@nat@height>\textheight\textheight\else\Gin@nat@height\fi}
\makeatother
% Scale images if necessary, so that they will not overflow the page
% margins by default, and it is still possible to overwrite the defaults
% using explicit options in \includegraphics[width, height, ...]{}
\setkeys{Gin}{width=\maxwidth,height=\maxheight,keepaspectratio}
% Set default figure placement to htbp
\makeatletter
\def\fps@figure{htbp}
\makeatother

\makeatletter
\makeatother
\makeatletter
\makeatother
\makeatletter
\@ifpackageloaded{caption}{}{\usepackage{caption}}
\AtBeginDocument{%
\ifdefined\contentsname
  \renewcommand*\contentsname{Table of contents}
\else
  \newcommand\contentsname{Table of contents}
\fi
\ifdefined\listfigurename
  \renewcommand*\listfigurename{List of Figures}
\else
  \newcommand\listfigurename{List of Figures}
\fi
\ifdefined\listtablename
  \renewcommand*\listtablename{List of Tables}
\else
  \newcommand\listtablename{List of Tables}
\fi
\ifdefined\figurename
  \renewcommand*\figurename{Figure}
\else
  \newcommand\figurename{Figure}
\fi
\ifdefined\tablename
  \renewcommand*\tablename{Table}
\else
  \newcommand\tablename{Table}
\fi
}
\@ifpackageloaded{float}{}{\usepackage{float}}
\floatstyle{ruled}
\@ifundefined{c@chapter}{\newfloat{codelisting}{h}{lop}}{\newfloat{codelisting}{h}{lop}[chapter]}
\floatname{codelisting}{Listing}
\newcommand*\listoflistings{\listof{codelisting}{List of Listings}}
\makeatother
\makeatletter
\@ifpackageloaded{caption}{}{\usepackage{caption}}
\@ifpackageloaded{subcaption}{}{\usepackage{subcaption}}
\makeatother
\makeatletter
\@ifpackageloaded{tcolorbox}{}{\usepackage[many]{tcolorbox}}
\makeatother
\makeatletter
\@ifundefined{shadecolor}{\definecolor{shadecolor}{rgb}{.97, .97, .97}}
\makeatother
\makeatletter
\makeatother
\ifLuaTeX
  \usepackage{selnolig}  % disable illegal ligatures
\fi
\usepackage[]{natbib}
\bibliographystyle{elsarticle-num}
\IfFileExists{bookmark.sty}{\usepackage{bookmark}}{\usepackage{hyperref}}
\IfFileExists{xurl.sty}{\usepackage{xurl}}{} % add URL line breaks if available
\urlstyle{same} % disable monospaced font for URLs
\hypersetup{
  pdftitle={Multi-material distributed recycling via Fused granular fabrication: rHDPE and rPET case of study},
  pdfauthor={Catalina Suescun; Bob Security; Cat Memes; Derek Zoolander},
  pdfkeywords={keyword1, keyword2},
  colorlinks=true,
  linkcolor={Blue},
  filecolor={Maroon},
  citecolor={Blue},
  urlcolor={Blue},
  pdfcreator={LaTeX via pandoc}}

\setlength{\parindent}{6pt}
\begin{document}

\begin{frontmatter}
\title{Multi-material distributed recycling via Fused granular
fabrication: rHDPE and rPET case of study}
\author[1]{Catalina Suescun%
\corref{cor1}%
\fnref{fn1}}
 \ead{alice@example.com} 
\author[2]{Bob Security%
%
\fnref{fn2}}
 \ead{bob@example.com} 
\author[2]{Cat Memes%
%
\fnref{fn3}}
 \ead{cat@example.com} 
\author[]{Derek Zoolander%
%
}
 \ead{derek@example.com} 

\affiliation[1]{organization={Université de Lorraine, Department
Name},addressline={Street Address},city={City},postcode={Postal
Code},postcodesep={}}
\affiliation[2]{organization={Another University, Department
Name},addressline={Street Address},city={City},postcode={Postal
Code},postcodesep={}}

\cortext[cor1]{Corresponding author}
\fntext[fn1]{This is the first author footnote.}
\fntext[fn2]{Another author footnote, this is a very long footnote and
it should be a really long footnote. But this footnote is not yet
sufficiently long enough to make two lines of footnote text.}
\fntext[fn3]{Yet another author footnote.}

        
\begin{abstract}
The high volume of plastic waste and the extremely low recycling rate
has created a serious challenge worldwide. Local distributed recycling
coupled to additive manufacturing (DRAM) offers a solution by
economically incentivizing local recycling. A new DRAM technology
capable of processing large quantities of plastic waste quickly is fused
granular fabrication (FGF), where solid shredded plastic waste can be
reused directly as 3D printing feedstock. This study presents an
experimental assessment of multi-material recycling printability, using
two of the most common thermoplastics in the beverage industry
polyethylene terephthalate (PET) and high-density polyethylene (HDPE)
and the feasibility of mixing PET and HDPE to be used as a feedstock
material for large-scale 3-D printing. After the material collection,
shredding, and cleaning its characterization, and optimization of
parameters for 3D printing was performed. Results showed the feasibility
of printing a large object from rPET/rHDPE flakes reducing the
production cost up to 88\%. .
\end{abstract}





\begin{keyword}
    keyword1 \sep 
    keyword2
\end{keyword}
\end{frontmatter}
    \ifdefined\Shaded\renewenvironment{Shaded}{\begin{tcolorbox}[enhanced, borderline west={3pt}{0pt}{shadecolor}, boxrule=0pt, frame hidden, interior hidden, breakable, sharp corners]}{\end{tcolorbox}}\fi

\hypertarget{introduction}{%
\section{Introduction}\label{introduction}}

The disposal of plastic waste is one of the most challenging current
environmental concerns given its systemic complexity \citep{evode2021}.
The mass of micro- / meso- plastics in the oceans are expected to exceed
the mass of the global stock of fish by 2050 \citep{macarthur2017}. More
critically, the global plastic annual production is expected to reach
1100 metric tons by the same year \citep{geyer2020}. The societal
awareness on plastic recycling have received substantial attention by
scientific, policymaker and general public \citep{soares2021}.
Unfortunately, the statistical analysis on the centralized recycling
process proves that it has been largely ineffective
\citep{godswillImpactsPlasticPollution2019a} as only 9\% of the plastic
that has been produced has been recycled from the total stock produced
since 1950 \citep{Geyer2017}. Therefore, it remains an open challenge to
identify alternatives to valorize discarded plastic material.

Distributed recycling and additive manufacturing (DRAM), is an
innovative technical approach to recycle plastic wastes
\citep{cruzsanchez2020, dertinger2020}. DRAM was first practiced with
recyclebots, which are waste plastic extruders that made filament for
conventional fused filament-based 3-D printers
\citep{baechler2013, zhong2018, woern2018}. Past research demonstrated
that using distributed recycling fits into the circular economy paradigm
\citep{Ford2016, Despeisse2016}; where consumers directly recycle their
own waste into consumer products from open source designs, from toys for
children \citep{Petersen2017} to adaptive aids for those with arthritis
\citep{gallup2018}. Distributed manufacturing is now in wide use
\citep{pearce2022}. In this way DRAM-based recycling is done in a closed
loop supply chain network \citep{santander2020}. This type of recycling
aims to reduce the environmental impact by the reduction of the
transportation from the waste source to recycling facilities
\citep{kreiger2014}. In that sense, it aims to propose innovative
closed-loop strategies using waste materials as raw resources
\citep{romani2021}.

Fused filament fabrication (FFF, which is also known as Fused Deposition
Modelling --FDM©-) is the most-widespread and established
extrusion-based AM technology due to the open source proliferation from
the self-replicating rapid prototyper (RepRap) project
\citep{jones2011, sells2009, bowyer2014}. This is due to its simplicity,
versatility, low-cost, and ability in the construction of geometrically
complex objects in the industrial and prosumer domains
\citep{romani2022}. Indeed, the open-source 3-D approach for 3-D
printers has enabled the technology to evolve in a radical manner for
manufacturing and prototyping adding value to the recycled material
\citep{cruzsanchez2020}. There are large efforts to find sustainable
feedstocks for 3-D printing \citep[\citep{Pakkanen2017}]{rett2021}.
Several studies in the literature have increase the spectrum of recycled
filament materials such as PLA \citep{cruzsanchez2017, anderson2017},
ABS \citep{mohammed2017a, mohammed2017}, PET
\citep{zander2018, vaucher2022}, HDPE
\citep{chong2017, mohammed2017a, baechler2013} PC \citep{gaikwad2018}.
In fact, using a comparative life cycle assessment in a low density
population case study of Michigan, USA, \citep{Kreiger2014} argued that
about of 100 billion MJ of energy per year could be saved in a
distributed approach, for the 984 million pounds of HDPE that are
recycled in the U.S. There is thus considerable evidence that DRAM can
reduce the energy consumption and greenhouse gases of the manufacturing
processes.

Most DRAM studies have been using mono-material for the fabrication of
feedstock for FFF. There are, however, several examples of mixed
materials including wood waste and recycled plastic
\citep{pringle2018, loschke2019} and textile fibers and recycled plastic
\citep{carrete2021}. Recently, \citep{zander2019b} reported the
manufacturing of composite filament from recycled PET/PP and PS/PP
blending through compatibilizer copolymer such as SEBS. Their results
revealed the technical printability of polypropylene blend composite
filaments from a thermo-mechanical characterization perspective.
Increasing the performance window of blending materials by
compatibilization which could be a relevant path for recycling plastics
in a local level and isolated areas contexts (e.g.~during humanitarian
crises \citep[\citep{corsini2022},\citep{lipsky2019}]{savonen2018},
supply chain disruptions \citep[\citep{choong202}, \citep{salmi2020},
\citep{attaran2020}]{novak2020} and/or isolated off-grid situations
using solar-powered 3-D printers \citep[\citep{gwamuri2016},
\citep{wong2015}, \citep{mohammed2018}]{king2014}). Likewise,
\citep{vaucher2022} studied the evaluation of the microstructure,
mechanical performance, and printing quality of filaments made from rPET
and rHDPE varying the wt\% of HDPE material from 0 to 10\%. They
confirmed the increase in the Young's modulus from 1.7 GPa of the pure
PET to 2.1 GPa for all the HDPE concentrations. Additionally, the
maximum stress of the bends were augmented with high HDPE
concentrations. Values were lower than virgin PET filament, yet similar
to commercial recycle ones. The addition of rHDPE at higher levels,
however, helped to meet the brittle-ductile transition in 15\% despite
the low interfacial tension of both polymers, allowing the printing of
quality parts.

While former studies have proven been successful in FFF, a new approach
to DRAM is fused granular fabrication (FGF) or fused particle
fabrication (FPF), where the material-extrusion AM systems print
directly from pellets, granules, flakes, shred or grinder material
\citep{fontana2022, woern2018}. In the context of recycling, this could
reduce the number of melt/extrusion cycles that degrade the material
needed in the filament fabrication process \citep{cruzsanchez2017a}. The
FGF technique opens up the potential of use recycle materials as well as
print large-scale objects either with a conventional cartesian 3-D
printer \citep{woern2018}, delta 3-D printer \citep{grassi2019} or
hangprinter \citep[\citep{rattan2023}]{petsiuk2022}. Research groups
corroborate that plastic waste can be used as feedstock materials for
FGF/FPF. \citep{alexandre2020} assessed the technical and economical
dimensions of virgin and shredded PLA printed in a self-modified FGF
machine and compared with FFF. The investigation showed that the use of
FGF reduced printing cost, time and its mechanical performance was
comparable with the obtained using the traditional FFF technique.
Likewise, \citep{woern2018} found comparable properties between PLA,
ABS, PP, and PET recycled and virgin materials. Later publications
demonstrated the technical and economic feasibility through the printing
of complex objects validating the possibility of recycle plastic with
FGF in both conventional and common FFF materials \citep{byard2019}, but
also recycle PC \citep{reich2019b} and rPET \citep{little2020}. Few
researchers, however, have addressed the problem of the direct printing
of recycled multi-materials, which might be a key step forward needed to
facilitate the ease of sorting and recycling post-consumer plastic waste
materials.

This study explores the potential of direct 3-D printing two immiscible
polymers commonly used in the beverage sector through a distributed
recycling process for its easily implementation operation at the local
level. To demonstrate the feasibility of the process, the bottled water
plastic most used in France of roughly 90\% of PET (body of the bottle)
and 10\% of HDPE (cap) now called \emph{rPET90//rHDPE10}, is used as a
test material. The experimental process of collection, characterization,
and printing of the recycled material is described and the results are
discussed in the context of widespread DRAM adoption at the community
level.


\renewcommand\refname{References}
  \bibliography{assets/Catalina.bib,assets/References.bib}


\end{document}
